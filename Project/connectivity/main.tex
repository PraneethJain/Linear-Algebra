\documentclass[12pt, a4paper]{article}
% \usepackage{graphicx}
% \usepackage[export]{adjustbox}
% \usepackage{amsmath}
% \graphicspath{{images/}}
\usepackage{titling}
\renewcommand\maketitlehooka{\null\mbox{}\vfill}
\renewcommand\maketitlehookd{\vfill\null}

\title{Analysis of Adjacency Matrices by Spectral Graph Theory}
\author{
  Name1 \\
  RollNum1
  \and
  Name2 \\
  RollNum2
  \and
  Name3 \\
  RollNum3
}
\date{MONSOON SEMESTER 2023}
\begin{document}
    \begin{titlepage}
      \maketitle
      \thispagestyle{empty}
    \end{titlepage}

    \section{Introduction}

      \subsection{Eigenvalue and Eigenvector}
      For a matrix $A$, a vector which when multiplied by $A$ gives a scalar multiple of itself is called an eigenvector of $A$. This scalar multiple is called the corresponding eigenvalue.

      $$Ax = \lambda x$$

      $x$ is an eigenvector and $\lambda$ is the corresponding eigenvalue

      \subsection{Graph}
      A graph is a set of vertices $V$ connected by a set of edges $E$.  
      $$G = (V, E)$$
        \subsubsection{Directed Graph}
        A directed graph is a graph where each edge represents an ordered pair of vertices
        \subsubsection{Undirected Graph}
        An undirected graph is a graph where each edge represents an unordered pair of vertices

        \subsubsection{Path}
        A sequence of vertices $$V_1V_2V_3...V_n$$ is a path if 
        $$\forall i \in \{1,2,3,..,n-1\} \qquad (V_{i}, V_{i+1}) \in E$$

        The length of such a path is $n-1$

      \subsection{Adjacency Matrix of a graph}
      For a graph $G$ with $n$ vertices, its adjacency matrix $A(G)$ is an $n$ by $n$ square matrix with $a_{i,j}$ equal to the number of edges connecting $V_i$ to $V_j$

        \subsubsection{Properties}
        For an undirected graph $G$
        \begin{itemize}
          \item $A(G)$ is symmetric 
          \item $A(G)$ has real eigenvalues
          \item $(i,j)$ entry of $A(G)^k$ represents the number of paths of length k from $V_i$ to $V_j$ $\quad \forall k \ge 1$
        \end{itemize}

    \section{Connectivity}
    We will analyse the connectivity of a graph using its adjacency matrix and its eigenvalues.

    \subsection{Connected Components \cite{baeldung}}
    \begin{flushleft}
    Consider a regular undirected graph $G$ (each vertex having degree $d$), and some eigenvector $x$ of $A(G)$ with its corresponding eigenvalue $\lambda$

    $$A(G)*x = \lambda*x$$
    $$\lambda x_i = \sum_{j} A_{ij}x_j$$
    $$|\lambda||x_i| \le \sum_{j} |A_{ij}||x_i|$$
    $$ |\lambda||x_i| \le d|x_i| \qquad (Since \quad |x_j| \le |x_i| \forall j)$$
    $$|\lambda| \le d$$

    All the eigenvalues of this graph are smaller than or equal to $d$. \cite{bhaskara4}

    Upon setting all $x_i$ to 1, we get $Ax = dx$, which gives
    $$\lambda_{max} = d$$

      The multiplicity of the maximum eigenvalue (equal to $d$) gives the number of connected components of $G$. \cite{bhaskara5} The number of connected components of a graph have various applications in different fields

      \begin{itemize}
        \item Social Network Analysis: Connected components are used to identify a set of individuals as a community. By finding these components, insights on information flow and social interactions are studied by researchers.
        \item Image Processing: Segmentation of objects and finding regions of interest is done by finding the connected components of an image. The image is converted into a graph, and then pixels sharing similar characteristics are grouped together. This is useful for object recognition and feature extraction.

        \item Network Routing: In large-scale computer networks, connected components are used by routing algorithms to determine the best path that the data packets should take in order to reach their destination.
      \end{itemize}
    \end{flushleft}

    \subsection{Number of paths in a graph \cite{mitpaths}}
    \begin{flushleft}
    Consider an undirected graph $G$ with adjacency matrix $A(G)$. Let $\lambda_1,\lambda_2,\lambda_3,...,\lambda_n$ be the eigenvalues of $A(G)$
      $$\forall i,j \quad \exists c_1,c_2,c_3,...,c_n \quad (A(G)^k)_{ij} = c_1\lambda_1^k + c_2\lambda_2^k + c_3\lambda_3^k + ... + c_n\lambda_n^k \quad \forall k \ge 1$$
      For each index $(i, j)$, upon identifying its corresponding $c_i$ values, the number of paths from $V_i$ to $V_j$ of any length can be calculated in constant time using this equation.

      Similarly, the total number of self loops of length k is given by
      $$\lambda_1^k + \lambda_2^k + \lambda_3^k + ... + \lambda_n^k$$

      The number of paths of given length between any two vertices is a very useful metric to have
      \begin{itemize}
        \item Network Analysis: The number of paths between any two access points in a network informs us of its reliability. We can figure out the number of alternate paths the network can take, and how to distribute the packets between all the paths in order to optimally transmit data.

        \item Reachability Analysis: The presence of a direct or indirect path between any two vertices gives insights on information flow and communication possibilities. The number of paths tells us how easily accessible any vertex is from any other vertex.

        \item Disease Spread: While modelling the spread of a disease, individuals and locations act as vertices in the graph, and the number of paths quantify how likely it is that the disease spread to the other end of that path. This informs us regarding control strategies and which locations require quarantine measures the most.
      \end{itemize}
    \end{flushleft} 

    \begin{thebibliography}{9}
    \bibitem{baeldung}
    Science, B. O. C., Science, B. O. C. (2022). Connected Components in a Graph | Baeldung on Computer Science. Baeldung on Computer Science. https://www.baeldung.com/cs/graph-connected-components
    \bibitem{bhaskara4}
    Aditya Bhaskara, CS Utah (2016). Graph Partitioning, basic linear algebra https://users.cs.utah.edu/~bhaskara/courses/x968/notes/lec4.pdf
    \bibitem{bhaskara5}
    Aditya Bhaskara, CS Utah (2016). Eigenvalues of $A_G$ and the Laplacion https://users.cs.utah.edu/~bhaskara/courses/x968/notes/lec5.pdf

    \bibitem{mitpaths}
    Derek Chen, Advay Goel (2021). Spectral Graph Theory https://math.mit.edu/research/highschool/primes/materials/2021/December/Chen-Goel.pdf

    \end{thebibliography}
\end{document}
